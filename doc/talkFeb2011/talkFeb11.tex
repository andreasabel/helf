\documentclass[12pt,pdflatex,hyperref={pdfkeywords={OrderedRepresentation Lambda},pdfstartview=Fit,bookmarks=true,bookmarksopen=true,pdfpagemode=None,colorlinks=true,linkcolor=unserblau,urlcolor=unserblau},notes=hide,t]{beamer}


\usepackage[latin1]{inputenc}
%\usepackage{beamerthemelined}
\usetheme[width=2cm]{Goettingen}
\usecolortheme{sidebartab}
\setbeamercovered{highly dynamic} 
\usepackage{xspace}
\usepackage{ifthen}
\usepackage{amsmath}
\usepackage{amssymb}
\usepackage{stmaryrd}

\usepackage[english]{babel}
%\usepackage[T1]{fontenc}
\usepackage{graphicx}
\usepackage{multimedia}

\definecolor{unserblau}{rgb}{0.21, 0.21, 0.70}

\usepackage{booktabs}
\usepackage[nobibnewpage,notocbib]{apacite}
\usepackage[absolute,overlay,quiet]{textpos}
\usepackage{ngerman}



% own
\newcommand{\la}{\lambda}
\newcommand{\ovlam}[1]{\mathcal{\lambda}^{#1}}
\newcommand{\be}{\beta}
\newcommand{\pa}[1]{\left( #1 \right)}
\newcommand{\sub}[2]{\left[ #1 / #2 \right]}
\newcommand{\subs}[4]{\left[ #1 / #2 , #3 / #4 \right]}
\newcommand{\bere}{\longrightarrow_\be}
\newcommand{\re}{\longrightarrow}

\newcommand{\type}{\mathsf{Type}}
\newcommand{\kind}{\mathsf{Kind}}
\newcommand{\sort}{\mathsf{Sort}}
\newcommand{\head}{\mathsf{Head}}
\newcommand{\ovpi}{\Pi}
\newcommand{\ovar}{\mathord{\bullet}}
\newcommand{\oapp}[1]{\,\oann{#1}}
\newcommand{\olam}[1]{\lambda^{#1}.\,}
\newcommand{\opi}[2]{\Pi\,#1\,^{#2}}
\newcommand{\oann}[1]{{}^{#1}\kern-0.15ex}

\newcommand{\ev}[3]{\llbracket{#1}\rrbracket^{#2}_{#3}}
\newcommand{\ap}{\lspace @ \lspace} % !! todo.

\newcommand{\sspace}{\,}
\newcommand{\lspace}{\ \,}




% \author[H. Wurst \& V. Egal]{Hans Wurst \and V�llig Egal \\
% 			    \href{mailto:(wurst|egal)@informatik.hu-berlin.de}{\texttt{(wurst|egal)@informatik.hu-berlin.de}}
% }
\author[Nicolai Kraus]{Nicolai Kraus}

\institute{Institut f�r Informatik\\ Ludwig-Maximilians-Universit�t M�nchen}
\date[2011-02-23]{23.02.2011}
\title{A Lambda Term Representation Based on Ordered Linear Logic}
\subtitle[TCS Oberseminar]{}

% \pgfdeclareimage[width=2cm]{logo}{hulogo} \logo{\pgfuseimage{logo}}


\addtobeamertemplate{sidebar right}{}{\vspace{-2cm}\insertlogo}
\addtobeamertemplate{footline}{}{\vspace{-1cm}\hskip2pt(\insertframenumber/\inserttotalframenumber) \insertshortsubtitle{} -- \insertshortdate\vskip2.2pt}
\setbeamercolor{footline}{fg=unserblau}
%\setbeamercolor*{alerted text}{fg=unserblau}

%\includeonlyframes{current}

\begin{document}
\frame{\titlepage}

% \section*{Overview}
% \frame[squeeze]{\frametitle{Overview}\tableofcontents}



\section{Motivation}

% spaces: \, \; \  \quad \qquad
\frame{
\frametitle{Motivation}
\[
\begin{array}{lr@{}l}
&(\la x.\la y. \sspace a \sspace b \sspace y) \sspace & g \sspace f \\
\\
\visible<2->{
\re &(\la y.\sspace a\sspace b \sspace y) & \sub g x \lspace f} \\
\\
\visible<3->{
\re &      (a\sspace b\sspace y) & \subs g x f y} \\
\\
\visible<4->{
\re &      \multicolumn 2 l {(a\sspace b)\subs g x f y \lspace y\subs g x f y}} \\
\\
\visible<5->{
\re &      \multicolumn 2 l {a\subs g x f y \lspace b\subs g x f y \lspace f} } \\
\\
\visible<6->{
% \re &      a \, b \, f }
\re &      \multicolumn 2 {c} {a \sspace b \sspace f} } \\
\end{array}
\]
}

% \frame{
% \frametitle{Motivation}
% \itemize{
% \item A
% \item b
% \item c
% }
% }

% \frame{
% \frametitle{De Bruijn Indices}
% \[
% \begin{array}{lr@{}l}
% &(\la x.\la y.aby) & gf \\ \\ \\
% &(\la \la ab0) & gf
% \end{array}
% \]
% }
\frame{
\frametitle{De Bruijn Indices}
\[
\begin{array}{r}
\la x. \la y. \la z. \sspace x \sspace z \sspace \pa{y \sspace z} \\ \\ \\
\la. \la. \la.  \sspace 2 \sspace 0 \sspace \pa{1 \sspace 0}
\end{array}
\]
}

\section{Ordered Representation}
\subsection{Ordered Terms}
\frame{
\frametitle{Ordered Terms}
\[
\begin{array}{lrlll}
t,u & ::= 	\visible<2->{&       & x & \mbox{free variable (named $x$)} } \\
			\visible<3->{&& \mid & \ovar & \mbox{bound variable (nameless)} } \\
			\visible<4->{&& \mid & t \oapp k u & \mbox{application} } \\
			\visible<5->{&& \mid & \olam {\vec k} t & \mbox{abstraction} } \\
% 			&& \mid & \opi A k B  & \mbox{function type}  ??
\end{array}
\]
\\
\begin{flushright}$\visible<5->{\vec k = (k_1, k_2, \ldots, k_n)}$\end{flushright}
}
\frame{
\frametitle{Example}
\[
\begin{array}{r}
\la x. \la y. \la z. \sspace x \sspace z \sspace \pa{y \sspace z} \\ \\ \\
\la. \la. \la.  \sspace 2 \sspace 0 \sspace \pa{1 \sspace 0} \\ \\ \\
\olam {\pa 0} \olam {\pa 1} \olam {\pa{1,1}} \ovar \oapp 1 \ovar \oapp 2 {\pa{\ovar \oapp 1 \ovar}}
\end{array}
\]
}
\frame{
\frametitle{Evaluation (example)}
\[
\begin{array}{lr@{}l}
&(\olam {\pa 0} \olam {\pa 1} \olam {\pa{1,1}} \ovar \oapp 1 \ovar \oapp 2 {\pa{\ovar \oapp 1 \ovar}}) &  \sspace g \sspace f \sspace n \\
\\
\visible<2->{
\re &(\olam {\pa 1} \olam {\pa{1,1}} \ovar \oapp 1 \ovar \oapp 2 {\pa{\ovar \oapp 1 \ovar}}) & \left[ g \right] \lspace f \sspace n} \\
\\
\visible<3->{
\re &      (\olam {\pa{1,1}} \ovar \oapp 1 \ovar \oapp 2 {\pa{\ovar \oapp 1 \ovar}}) & \left[ g, f \right] \lspace n} \\
\\
\visible<4->{
\re &      (\ovar \oapp 1 \ovar \oapp 2 {\pa{\ovar \oapp 1 \ovar}}) & \left[ g, n, f, n \right]} \\
\\
\visible<5->{
\re &      (\ovar \oapp 1 \ovar)\left[ g, n \right] \lspace (\ovar \oapp 1 \ovar)&\left[ f, n \right]                } \\
\\
\visible<6->{
\re &      \ovar[g]  \lspace \ovar[n]  \lspace (\ovar[f]  \lspace \ovar[n]) }& \\
\\
\visible<7->{
\re &      g \sspace n  \sspace (f \sspace n) & } 
\end{array}
\]
}

\subsection{Ordered Values}
\frame{
\frametitle{Ordered Values}
\[ 
\begin{array}{lrlll}
v,w & ::= 	\visible<2->{& h \vec v & \mbox{large application}  \\ 
&&& \mbox{($h$ is a free variable)} \\ }
\\
% 	\visible<3->{& \mid 	& \left( \ovlam {\vec k} t\right) \left[ \vec u \right] & \mbox{closure} } % !! really "closure" ??
	\visible<3->{& \mid 	& \left( \ovlam {\vec k}
            t\right)^{\mbox{$\eta$}}_{\mbox{$\vec v$}} & \mbox{closure} } % !! really "closure" ??
\end{array}
\]
}

\subsection{Ordered Evaluation}
\frame{
\frametitle{Ordered Evaluation}
\[ 
\begin{array}{llll}
& \ev x \varrho {()} & = & \varrho(x) 
\\[1em]
\visible<2->{ & \ev \ovar \varrho {(v_1)} & = & v_1 }
\\[1em]
\visible<3->{ & \ev {\olam {\vec k} t}  \varrho {\vec v} & = & \pa{ \ovlam {\vec k} t}_{ \vec v }^\varrho }
\\[1em]
\visible<4->{ & \ev {t \oapp k u}  \varrho {(v_1, \ldots, v_n)} & = & {\ev t  \varrho {(v_1, \ldots, v_{n-k})}} \ap {\ev u  \varrho {(v_{n-k+1}, \ldots, v_n)}} }
\\[2em]
\visible<5->{ & \pa{h \vec v} \ap w & = & h \sspace \sspace (\vec v, w) }
\\[1em]
\visible<6->{ & \left( \ovlam {\vec k} t\right)_{ \vec v }^\eta  \ap  w & = & \ev t \eta {\vec {v'}}  \\
\multicolumn 2 r {\mbox{where } \vec v} & = & (v_1, \ldots, v_n) \\
\multicolumn 2 r {\vec {v'}} & = & (v_1, \ldots, v_{k_1}, w, v_{k_1+1}, \ldots, v_{k_1+k_2}, w, \ldots ) }
% && \mbox{where } & \vec u = (u_1, \ldots, u_n) \\
% &&               & \vec {u'} =  (u_1, \ldots, u_{k_1}, w, u_{k_1+1}, \ldots, u_{k_1+k_2}, w, \ldots ) }
\end{array}
\]
}


\subsection{Extension}
\frame{
\frametitle{Ordered Terms (Extension)}
\[
\begin{array}{lrlll}
t,u, A, B & ::= 	&       & x & \mbox{free variable (named $x$)}  \\
			&& \mid & \ovar & \mbox{bound variable (nameless)}  \\
			&& \mid & t \oapp k u & \mbox{application}  \\
			&& \mid & \olam {\vec k} t & \mbox{abstraction}  \\
\visible<2->{&& \mid & \opi A k B  & \mbox{function type}  } \\
\visible<3->{&& \mid & \mathsf{constant}   }\\
\visible<3->{&& \mid & \mathsf{definition}   }\\
\visible<3->{&& \mid & \mathsf{Sort}   }
\end{array}
\]
}

\frame{
\frametitle{Ordered Values (Extension)}
\[ 
\begin{array}{lrlll}
v,w,V,W & ::= 	\visible<1->{& h \vec v & \mbox{large application}  \\ 
&&& \mbox{($h$ is a free variable)} \\ }
\\
% 	\visible<3->{& \mid 	& \left( \ovlam {\vec k} t\right) \left[ \vec u \right] & \mbox{closure} } % !! really "closure" ??
	\visible<1->{& \mid 	& \left( \ovlam {\vec k} t\right)^\eta_{\vec v} & \mbox{closure} } \\
	\\
	\visible<2->{& \mid 	& \left( \la x.v\right)^\eta & \mbox{abstraction} } \\
	\\
	\visible<3->{& \mid 	& \ovpi V W & \mbox{function type} } \\
	\\
	\visible<4->{& \mid 	& \mathsf{Sort} & \mbox{} } 
\end{array}
\]
}

\frame{
\frametitle{Ordered Evaluation (Extension)}
\[ 
\begin{array}{llll}

\visible<1->{& \ev {\opi A k B}  \varrho {(v_1, \ldots, v_n)} & = & \ovpi \hspace{1ex} \ev{A} \varrho {(v_1, \ldots, v_{n-k})} \ev{B} \varrho {(v_{n-k+1}, \ldots, v_n)} } 
\\[1em]
\visible<2->{& \left( \la x.v\right)^\eta \ap w & = & \llparenthesis v \rrparenthesis^{\eta [x \mapsto w]}} 
\\[1em]
\\
\visible<3->{& \llparenthesis \left( \la y.v'\right)^{\eta'} \rrparenthesis^\eta   & = &  \left( \la y.v'\right)^{\eta''}  } 
\\[1em]
\visible<4->{& \llparenthesis \left( \ovlam {\vec k} t\right)_{ \vec v }^{\eta'} \rrparenthesis^\eta   & = &  
\left( \ovlam {\vec k} t\right)_{ \llparenthesis \vec v \rrparenthesis^\eta }^{\eta''} }
\\[1em]
\visible<5->{& \llparenthesis h \vec v \rrparenthesis^\eta   & = &
  \mbox{apply } \eta(h) \mbox{ to } \llparenthesis \vec v
  \rrparenthesis ^ \eta} \\
\end{array}
\]
\\
\[ \visible<3->{
\eta''(x) = 
\begin{cases} 
\eta(x) & \text{if possible, i.e. if $x$ is mentioned in $\eta$}
\\
\llparenthesis \eta'(x) \rrparenthesis^\eta & \text{otherwise.}
\end{cases}
}
\]
}












\section{Simple Closures}

\frame{
% \frametitle{Simple Closures}
Basic syntax:
\[
\begin{array}{lrll} % KIND / TYPE
t,u,A,B & ::= & x & \mbox{variable (named $x$)} \\
			                   & \mid & t u & \mbox{application} \\
			                   & \mid & \la x.t & \mbox{abstraction} \\
% 			                   & \mid & \Pi x^?:A.\,B  & \mbox{function type} \\
			                   & \mid & \Pi x:A.\,B  & \mbox{function type} \\
			                   & \mid & \mathsf{constant} & \\
			                   & \mid & \mathsf{definition} & 
\end{array}
\]
}


\subsection{Simple Closures - Values}
\frame{
\frametitle{Simple Closures - Values}


\[
\begin{array}{lrlll}
v,w,V,W & ::= 	\visible<1->{& h \sspace \vec v & \mbox{large application}  \\ 
% &&& \mbox{($h$ is a free variable, constant or definition)} \\ }
&&& \mbox{($h$ is a free variable, } \\
&&& \mbox{constant or definition)} \\ }
\\
% 	\visible<2->{& \mid 	& \left( \la x^?. t\right)^\eta & \mbox{closure} } \\
	\visible<2->{& \mid 	& \left( \la x. t\right)^\eta & \mbox{closure} } \\
	\\
	\visible<3->{& \mid 	& \left( \la x.v\right)^\eta & \mbox{abstraction} } \\
	\\
	\visible<4->{& \mid 	& \ovpi V W & \mbox{function type} } \\
	\\
	\visible<5->{& \mid 	& \mathsf{Sort} & \mbox{} } 
\end{array}
\]
}


\section{Beta Normal Values - Hereditary Substitution}
\frame{
\frametitle{Beta Normal Values - Hereditary Substitution}
\begin{itemize}
\item all values in beta normal form
\item uses Hereditary Substitution$^{(*)}$\\[2em]
\end{itemize}

% $^{(*)}$ K. Watkins, I. Cervesato, F. Pfenning, D. Walker\\
%  A Concurrent Logical Framework:
% The Propositional Fragment
$^{(*)}$ K. Watkins, I. Cervesato, F. Pfenning, D. Walker (2003). \emph{A Concurrent Logical Framework: The Propositional Fragment}
}

% data HVal
%   = HBound Int A.Name [HVal]       -- }   (bound head variable)
%   | HVar A.Name HVal  [HVal]       -- }
%   | HCon A.Name HVal  [HVal]       -- }-> Head
%   | HDef A.Name HVal HVal [HVal]   -- }
%   | HLam A.Name HVal
%   | HK HVal                        -- constant Lambda, not binding anything and therefore not counted by de Bruijn indices 
%   | HSort Value.Sort
%   | HFun HVal HVal
%   | HDontCare



\section{Comparison}
\frame{
\frametitle{Typechecking}
% \begin{itemize}
% \item comparison of the implementations:
% \item typechecking large Twelf-files using the 
% \end{itemize}
\begin{center}
comparison of the implementations:\\ 

typechecking large LF terms 
\end{center}
}

\frame{
\frametitle{Comparison - 6000.elf $^{(*)}$ (3.8 MB)}
\begin{tabular}{| l || c | c |}
\hline
& time (sec) & space (MB) \\
\hline
\hline
Ordered & 18.87 & 1111 \\
\hline
Simple Closures & 18.54 & 1152\\
\hline
Beta Normal Values & 27.62 & 2034\\
\hline
\end{tabular}
\\[2em]
baerentatze \\
7999 MiB, AMD Phenom II X4 B95
\\[2em]
$^{(*)}$ kindly provided by Andrew W. Appel, Princeton University
}


\frame{
\frametitle{Comparison - 10000.elf $^{(*)}$ (12.8 MB)}
\begin{tabular}{| l || c | c |}
\hline
& time (sec) & space (MB) \\
\hline
\hline
Ordered & 61.01 & 3230\\
\hline
Simple Closures & 59.96 & 3302\\
\hline
Beta Normal Values & 98.67 & 5878\\
\hline
\end{tabular}
\\[2em]
baerentatze \\
7999 MiB, AMD Phenom II X4 B95
\\[2em]
$^{(*)}$ kindly provided by Andrew W. Appel, Princeton University
}



\frame{
\frametitle{Comparison - 12000.elf $^{(*)}$ (17.8 MB)}
\begin{tabular}{| l || c | c |}
\hline
& time (sec) & space (MB) \\
\hline
\hline
Ordered & 84.31 & 5096 \\
\hline
Simple Closures & 83.63 & 5226 \\
\hline
Beta Normal Values & 137.73 & 8513 \\
\hline
\end{tabular}
\\[2em]
baerentatze \\
7999 MiB, AMD Phenom II X4 B95
\\[2em]
$^{(*)}$ kindly provided by Andrew W. Appel, Princeton University
}


\frame{
\frametitle{Comparison - w32\_sig\_semant.elf $^{(*)}$ (20.8 MB)}
\begin{tabular}{| l || c | c |}
\hline
& time (sec) & space (MB) \\
\hline
\hline
Ordered & 108.44 & 8877\\
\hline
Simple Closures & 94.34 & 5068 \\
\hline
Beta Normal Values & 169.76 & 9044  \\
\hline
(Twelf r1697) & (ca. 22) & (2720)  \\
\hline
\end{tabular}
\\[2em]
baerentatze \\
7999 MiB, AMD Phenom II X4 B95
\\[2em]
$^{(*)}$ kindly provided by Andrew W. Appel, Princeton University
}







% \frame{
% \frametitle{THE END}
% }



% \section{Literatur}
% \frame[squeeze]{
% 	\frametitle{Literatur}
% 	
% 	%\nocite{bla}
% 	%\bibliographystyle{apacite}
% 	%\bibliography{vortrag}
% }

\end{document}