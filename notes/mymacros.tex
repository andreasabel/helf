\usepackage{xspace}
\usepackage{ifthen}
\usepackage{amsmath}
\usepackage{amssymb}
%\usepackage{amsthm}
\usepackage{stmaryrd}
%\usepackage[all]{xy}
%\usepackage{txfonts} 

% \newtheorem{theorem}{Theorem}
%\newtheorem*{theorem*}{Theorem}
% \newtheorem{proposition}[theorem]{Proposition}
% \newtheorem{lemma}[theorem]{Lemma}
% \newtheorem{corollary}[theorem]{Corollary}
% \newtheorem{conjecture}[theorem]{Conjecture}
% \newtheorem{definition}[theorem]{Definition}
% \newtheorem{remark}[theorem]{Remark}
% \newtheorem{example}[theorem]{Example}

\newcommand{\oann}[1]{{}^{#1}\kern-0.15ex}
\newcommand{\ovar}{\mathord{\bullet}}
% \newcommand{\institute}[1]{}

% own
% general
\newcommand{\sspace}{\,}
\newcommand{\lspace}{\ \,}
\newcommand{\shspace}{\vspace{1ex}}

% also general
\newcommand{\type}{\mathsf{Type}}
\newcommand{\kind}{\mathsf{Kind}}
\newcommand{\sort}{\mathsf{Sort}}
\newcommand{\la}{\lambda}
\newcommand{\emptyVec}{[]}
\newcommand{\ve}[1]{[#1]}
\newcommand{\emphSec}[1]{#1}
\newcommand{\dom}[1]{\mathsf{dom}\left(#1\right)}

% abstract syntax:
\newcommand{\ApA}[2]{#1 \sspace #2}
\newcommand{\LaA}[2]{\la {#1}. \sspace #2}
\newcommand{\LaAT}[3]{\la {#1}:{#2}. \sspace #3}
\newcommand{\PiA}[3]{\Pi \sspace #1:#2 . \sspace #3}

% ordered terms: 
\newcommand{\ApO}[3]{#1 \sspace ^{#2} \sspace #3}
\newcommand{\oapp}[1]{\sspace ^{#1} \sspace}
\newcommand{\LaO}[2]{\la ^{#1} . \sspace #2}
\newcommand{\PiO}[3]{\Pi \sspace #1 \sspace ^{#2} \sspace #3}

% ordered values:
\newcommand{\ClosO}[4]{\left(\la^{#1} . \sspace {#2} \right)^{#3}_{#4}}
\newcommand{\AbsO}[3]{\left( \la {#1}. \sspace {#2} \right)^{#3}}
\newcommand{\FunO}[2]{\Pi \sspace #1 \sspace #2}

% ordered transformation:
\newcommand{\transO}[3]{ {#1}^{\mathcal{T}\left({#2}, \sspace {#3}\right)}}
\newcommand{\upch}[1]{\Uparrow^{#1}}
\newcommand{\result}[3]{\left( \, {#1}, \sspace {#2}, \sspace {#3} \, \right)}

% simple closures values:
\newcommand{\ClosS}[3]{\left(\la {#1}. \sspace #2 \right)^{#3}}
\newcommand{\KS}[2]{\left(\la . \sspace #1 \right)^{#2}}
\newcommand{\FunS}[2]{\Pi \sspace #1 \sspace #2}
\newcommand{\AbsS}[3]{\left( \la^{#1}. \sspace {#2} \right)^{#3}}

%  hereditary terms:
\newcommand{\LaH}[1]{\la . \sspace #1}
\newcommand{\KH}[1]{\la^c . \sspace #1}
\newcommand{\ApH}[2]{#1 \sspace #2}
\newcommand{\PiH}[2]{\Pi \sspace #1 \sspace #2}

% hereditary values:
\newcommand{\LaVH}[1]{\LaH #1} % same as the term
\newcommand{\KVH}[1]{\KH #1} % same
\newcommand{\FunH}[2]{\PiH #1 #2}

%hereditary subs:
% \newcommand{\hersub}[3]{{#1}\ll^{#2} {#3}}
\newcommand{\hersub}[3]{{#1}\,[{#3}/{#2}]}
\newcommand{\lift}[2]{ #2 \uparrow^{#1}}

% \newcommand{\ovlam}[1]{\mathcal{\lambda}^{#1}}
\newcommand{\be}{\beta}
\newcommand{\pa}[1]{\left( #1 \right)}
\newcommand{\sub}[2]{\left[ #1 / #2 \right]}
\newcommand{\subs}[4]{\left[ #1 / #2 , #3 / #4 \right]}
\newcommand{\multisubs}[4]{\left[ #1 / #2 , \ldots, #3 / #4 \right]}
\newcommand{\bere}{\longrightarrow_\be}
\newcommand{\re}{\longrightarrow}

\newcommand{\head}{\mathsf{Head}}

\newcommand{\ev}[3]{\llbracket{#1}\rrbracket^{#2}_{#3}}
\newcommand{\ap}{\,@\,} 
\newcommand{\valsub}[2]{\llparenthesis {\,#1\,} \rrparenthesis^{#2}} 
