\usepackage{xspace}
\usepackage{ifthen}
\usepackage{amsmath}
\usepackage{amsthm}
\usepackage{amssymb}
\usepackage{stmaryrd}
%\usepackage[all]{xy}
%\usepackage{txfonts} 

\usepackage{proof}

% \newtheorem{theorem}{Theorem}
%\newtheorem*{theorem*}{Theorem}
% \newtheorem{proposition}[theorem]{Proposition}
% \newtheorem{lemma}[theorem]{Lemma}
% \newtheorem{corollary}[theorem]{Corollary}
% \newtheorem{conjecture}[theorem]{Conjecture}
% \newtheorem{definition}[theorem]{Definition}
% \newtheorem{remark}[theorem]{Remark}
% \newtheorem{example}[theorem]{Example}

% Andreas: latin etc. abbrev
\newcommand{\abbrev}[1]{#1} % alternative: \emph{#1}
\newcommand{\cf}{\abbrev{cf.}\ }
\newcommand{\eg}{\abbrev{e.\,g.}}
\newcommand{\Eg}{\abbrev{E.\,g.}}
\newcommand{\ie}{\abbrev{i.\,e.}}
\newcommand{\Ie}{\abbrev{I.\,e.}}
\newcommand{\etal}{\abbrev{et.\,al.}}
\newcommand{\wwlog}{w.\,l.\,o.\,g.} % \wlog is ``write into log file''
\newcommand{\Wlog}{W.\,l.\,o.\,g.}
\newcommand{\wrt}{w.\,r.\,t.}

% Andreas: paragraphs
\newcommand{\para}[1]{\paragraph*{\it#1}}
\newcommand{\paradot}[1]{\para{#1.}}



\newtheorem{prop}{Proposition}
\newtheorem{defin}[prop]{Definition}
\newtheorem*{cor}{Corollary}


\newcommand{\oann}[1]{{}^{#1}\kern-0.15ex}
\newcommand{\ovar}{\mathord{\bullet}}
% \newcommand{\institute}[1]{}

% own
% general
\newcommand{\comment}[1]{} % hack, useful in some cases.
\newcommand{\sspace}{\,}
\newcommand{\lspace}{\ \,}
\newcommand{\shspace}{\vspace{1ex}}

% also general
\newcommand{\type}{\mathsf{Type}}
\newcommand{\kind}{\mathsf{Kind}}
\newcommand{\sort}{\mathsf{Sort}}
\newcommand{\la}{\lambda}
\newcommand{\emptyVec}{[]}
\newcommand{\emptySet}{\emptyset}
\newcommand{\ve}[1]{[#1]}
\newcommand{\emphSec}[1]{#1}
\newcommand{\dom}[1]{\mathsf{dom}\left(#1\right)}
\newcommand{\length}[1]{\mbox{length}(#1)}

% abstract syntax:
\newcommand{\ApA}[2]{#1 \sspace #2}
\newcommand{\LaA}[2]{\la {#1}. \sspace #2}
\newcommand{\LaAT}[3]{\la {#1}:{#2}. \sspace #3}
\newcommand{\PiA}[3]{\Pi \sspace #1:#2 . \sspace #3}

% ordered terms: 
\newcommand{\ApO}[3]{#1 \sspace ^{#2} \sspace #3}
\newcommand{\oapp}[1]{\sspace ^{#1} \sspace}
\newcommand{\LaO}[2]{\la ^{#1} . \sspace #2}
\newcommand{\PiO}[3]{\Pi \sspace #1 \sspace ^{#2} \sspace #3}
% free variables in a term:
\newcommand{\freevars}[1]{\mathsf{fV}(#1)}

% ordered values:
\newcommand{\ClosO}[4]{\left(\la^{#1} . \sspace {#2} \right)^{#3}_{#4}}
\newcommand{\AbsO}[3]{\left( \la {#1}. \sspace {#2} \right)^{#3}}
\newcommand{\FunO}[2]{\Pi \sspace #1 \sspace #2}
% insertion (like it is done for beta reduction):
\newcommand{\multiinsert}[3]{#1 ^ {\sspace #3 / #2}}

% simpyfied:
%\newcommand{\Val}[3]{\left(\la^{#1} . \sspace {#2} \right)^{#3}}
\newcommand{\Val}[3]{ (\la^{#1} . \sspace {#2} )^{#3}}

% ordered transformation:
\newcommand{\transO}[3]{ {#1}^{\mathcal{T}\left({#2}, \sspace {#3}\right)}}
\newcommand{\upch}[1]{\Uparrow^{#1}}
\newcommand{\result}[3]{\left( \, {#1}, \sspace {#2}, \sspace {#3} \, \right)}

% simple closures values:
\newcommand{\ClosS}[3]{\left(\la {#1}. \sspace #2 \right)^{#3}}
\newcommand{\KS}[2]{\left(\la . \sspace #1 \right)^{#2}}
\newcommand{\FunS}[2]{\Pi \sspace #1 \sspace #2}
\newcommand{\AbsS}[3]{\left( \la^{#1}. \sspace {#2} \right)^{#3}}

%  hereditary terms:
\newcommand{\LaH}[1]{\la . \sspace #1}
\newcommand{\KH}[1]{\la^c . \sspace #1}
\newcommand{\ApH}[2]{#1 \sspace #2}
\newcommand{\PiH}[2]{\Pi \sspace #1 \sspace #2}

% hereditary values:
\newcommand{\LaVH}[1]{\LaH #1} % same as the term
\newcommand{\KVH}[1]{\KH #1} % same
\newcommand{\FunH}[2]{\PiH #1 #2}

%hereditary subs:
% \newcommand{\hersub}[3]{{#1}\ll^{#2} {#3}}
\newcommand{\hersub}[3]{{#1}\,[{#3}/{#2}]}
\newcommand{\lift}[2]{ #2 \uparrow^{#1}}

% \newcommand{\ovlam}[1]{\mathcal{\lambda}^{#1}}
\newcommand{\be}{\beta}
\newcommand{\pa}[1]{\left( #1 \right)}
\newcommand{\sub}[2]{\left[ #1 / #2 \right]}
\newcommand{\subs}[4]{\left[ #1 / #2 , #3 / #4 \right]}
\newcommand{\multisubs}[4]{\left[ #1 / #2 , \ldots, #3 / #4 \right]}
\newcommand{\bere}{\longrightarrow_\be}
\newcommand{\re}{\longrightarrow}

\newcommand{\head}{\mathsf{Head}}

% \newcommand{\ev}[3]{\llbracket{#1}\rrbracket^{#2}_{#3}}
\newcommand{\ev}[2]{\llbracket{#1}\rrbracket_{#2}}
\newcommand{\ap}{\,@\,} 
\newcommand{\valsub}[2]{\llparenthesis {\,#1\,} \rrparenthesis^{#2}} 

% Parsing
% \newcommand{\lrhd}[1]{\mathrel{\mathord{\lhd}{^{#1}}\mathord{\rhd}}}
% \newcommand{\parse}[3]{#1 \, {\lrhd #2} \, #3 }
\newcommand{\parse}[3]{#1 \, {  \mathord{\lhd} {#2} \mathord{\rhd}  } \, #3 }
\newcommand{\parseFun}[1]{\left(\mathord{\lhd} {#1} \mathord{\rhd}\right)}
\newcommand{\parseFunE}{\mathfrak {parse } \,}                                                     % TODO !!
\newcommand{\print}[3]{#1 \, {  \mathord{\rhd} {#2} \mathord{\lhd}  } \, #3 }
\newcommand{\infru}[2]{\infer{#2}{#1}}
\newcommand{\infnamed}[3]{\infer[#1]{#3}{#2}}

\newcommand{\x}{X} % set of variables
\newcommand{\te}{T} % set of terms in basic syntax
\newcommand{\ot}{oT} % ordered terms
\newcommand{\otw}{oT_\Gamma} % ordered terms
\newcommand{\otc}{\overline{oT}} % closed ordered terms
\newcommand{\va}{V} % values
\newcommand{\xl}{\vec \x} % variable list
\newcommand{\vl}{\vec \va} % values list
\newcommand{\gl}{\vec \Gamma} % gamma list

% Printing / ``translating''
% \newcommand{\pr}[2]{ \prE  \hspace{-2pt} \left( #1 \ , \  #2 \right)}
\newcommand{\pr}[2]{ \prE   \left( #1 \ , \  #2 \right)}
\newcommand{\prE}{\lhd \hspace{-8pt} \lhd}
\newcommand{\prVal}[1]{\lhd \hspace{-2pt} \left( #1 \right)}
\newcommand{\prValE}{\mathord{\lhd}}
\newcommand{\prTree}[1]{\lhd \hspace{-8pt} \lhd \hspace{-8pt} \lhd \left( #1 \right)}
\newcommand{\prTreeE}{\lhd \hspace{-8pt} \lhd \hspace{-8pt} \lhd }

% one step reduction
\newcommand{\osr}{\rightarrow}
\newcommand{\tree}[2]{\langle \hspace{-2pt}\langle #1 \ , \  #2 \rangle \hspace{-2pt}\rangle}
%\newcommand{\tree}[2]{\left\langle \left\langle #1 \ , \  #2 \right\rangle \right\rangle}

\newcommand{\evTr}{\mathsf{evalTrees}}