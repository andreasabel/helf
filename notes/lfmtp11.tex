\nonstopmode
\documentclass[submission,copyright,creativecommons]{eptcs}
\providecommand{\event}{LFMTP'11} % Name of the event you are submitting to
% \usepackage{breakurl}             % Not needed if you use pdflatex only.
\usepackage{hyperref}
\usepackage{proof}
\usepackage[all]{xy}

\usepackage{xspace}
\usepackage{ifthen}
\usepackage{amsmath}
\usepackage{amsthm}
\usepackage{amssymb}
\usepackage{stmaryrd}
%\usepackage[all]{xy}
%\usepackage{txfonts} 

\usepackage{proof}

% \newtheorem{theorem}{Theorem}
%\newtheorem*{theorem*}{Theorem}
% \newtheorem{proposition}[theorem]{Proposition}
% \newtheorem{lemma}[theorem]{Lemma}
% \newtheorem{corollary}[theorem]{Corollary}
% \newtheorem{conjecture}[theorem]{Conjecture}
% \newtheorem{definition}[theorem]{Definition}
% \newtheorem{remark}[theorem]{Remark}
% \newtheorem{example}[theorem]{Example}

% Andreas: latin etc. abbrev
\newcommand{\abbrev}[1]{#1} % alternative: \emph{#1}
\newcommand{\cf}{\abbrev{cf.}\ }
\newcommand{\eg}{\abbrev{e.\,g.}}
\newcommand{\Eg}{\abbrev{E.\,g.}}
\newcommand{\ie}{\abbrev{i.\,e.}}
\newcommand{\Ie}{\abbrev{I.\,e.}}
\newcommand{\etal}{\abbrev{et.\,al.}}
\newcommand{\wwlog}{w.\,l.\,o.\,g.} % \wlog is ``write into log file''
\newcommand{\Wlog}{W.\,l.\,o.\,g.}
\newcommand{\wrt}{w.\,r.\,t.}

% Andreas: paragraphs
\newcommand{\para}[1]{\paragraph*{\it#1}}
\newcommand{\paradot}[1]{\para{#1.}}



\newtheorem{prop}{Proposition}
\newtheorem{defin}[prop]{Definition}
\newtheorem*{cor}{Corollary}


\newcommand{\oann}[1]{{}^{#1}\kern-0.15ex}
\newcommand{\ovar}{\mathord{\bullet}}
% \newcommand{\institute}[1]{}

% own
% general
\newcommand{\comment}[1]{} % hack, useful in some cases.
\newcommand{\sspace}{\,}
\newcommand{\lspace}{\ \,}
\newcommand{\shspace}{\vspace{1ex}}

% also general
\newcommand{\type}{\mathsf{Type}}
\newcommand{\kind}{\mathsf{Kind}}
\newcommand{\sort}{\mathsf{Sort}}
\newcommand{\la}{\lambda}
\newcommand{\emptyVec}{[]}
\newcommand{\emptySet}{\emptyset}
\newcommand{\ve}[1]{[#1]}
\newcommand{\emphSec}[1]{#1}
\newcommand{\dom}[1]{\mathsf{dom}\left(#1\right)}
\newcommand{\length}[1]{\mbox{length}(#1)}

% abstract syntax:
\newcommand{\ApA}[2]{#1 \sspace #2}
\newcommand{\LaA}[2]{\la {#1}. \sspace #2}
\newcommand{\LaAT}[3]{\la {#1}:{#2}. \sspace #3}
\newcommand{\PiA}[3]{\Pi \sspace #1:#2 . \sspace #3}

% ordered terms: 
\newcommand{\ApO}[3]{#1 \sspace ^{#2} \sspace #3}
\newcommand{\oapp}[1]{\sspace ^{#1} \sspace}
\newcommand{\LaO}[2]{\la ^{#1} . \sspace #2}
\newcommand{\PiO}[3]{\Pi \sspace #1 \sspace ^{#2} \sspace #3}
% free variables in a term:
\newcommand{\freevars}[1]{\mathsf{fV}(#1)}

% ordered values:
\newcommand{\ClosO}[4]{\left(\la^{#1} . \sspace {#2} \right)^{#3}_{#4}}
\newcommand{\AbsO}[3]{\left( \la {#1}. \sspace {#2} \right)^{#3}}
\newcommand{\FunO}[2]{\Pi \sspace #1 \sspace #2}
% insertion (like it is done for beta reduction):
\newcommand{\multiinsert}[3]{#1 ^ {\sspace #3 / #2}}

% simpyfied:
%\newcommand{\Val}[3]{\left(\la^{#1} . \sspace {#2} \right)^{#3}}
\newcommand{\Val}[3]{ (\la^{#1} . \sspace {#2} )^{#3}}

% ordered transformation:
\newcommand{\transO}[3]{ {#1}^{\mathcal{T}\left({#2}, \sspace {#3}\right)}}
\newcommand{\upch}[1]{\Uparrow^{#1}}
\newcommand{\result}[3]{\left( \, {#1}, \sspace {#2}, \sspace {#3} \, \right)}

% simple closures values:
\newcommand{\ClosS}[3]{\left(\la {#1}. \sspace #2 \right)^{#3}}
\newcommand{\KS}[2]{\left(\la . \sspace #1 \right)^{#2}}
\newcommand{\FunS}[2]{\Pi \sspace #1 \sspace #2}
\newcommand{\AbsS}[3]{\left( \la^{#1}. \sspace {#2} \right)^{#3}}

%  hereditary terms:
\newcommand{\LaH}[1]{\la . \sspace #1}
\newcommand{\KH}[1]{\la^c . \sspace #1}
\newcommand{\ApH}[2]{#1 \sspace #2}
\newcommand{\PiH}[2]{\Pi \sspace #1 \sspace #2}

% hereditary values:
\newcommand{\LaVH}[1]{\LaH #1} % same as the term
\newcommand{\KVH}[1]{\KH #1} % same
\newcommand{\FunH}[2]{\PiH #1 #2}

%hereditary subs:
% \newcommand{\hersub}[3]{{#1}\ll^{#2} {#3}}
\newcommand{\hersub}[3]{{#1}\,[{#3}/{#2}]}
\newcommand{\lift}[2]{ #2 \uparrow^{#1}}

% \newcommand{\ovlam}[1]{\mathcal{\lambda}^{#1}}
\newcommand{\be}{\beta}
\newcommand{\pa}[1]{\left( #1 \right)}
\newcommand{\sub}[2]{\left[ #1 / #2 \right]}
\newcommand{\subs}[4]{\left[ #1 / #2 , #3 / #4 \right]}
\newcommand{\multisubs}[4]{\left[ #1 / #2 , \ldots, #3 / #4 \right]}
\newcommand{\bere}{\longrightarrow_\be}
\newcommand{\re}{\longrightarrow}

\newcommand{\head}{\mathsf{Head}}

% \newcommand{\ev}[3]{\llbracket{#1}\rrbracket^{#2}_{#3}}
\newcommand{\ev}[2]{\llbracket{#1}\rrbracket_{#2}}
\newcommand{\ap}{\,@\,} 
\newcommand{\valsub}[2]{\llparenthesis {\,#1\,} \rrparenthesis^{#2}} 

% Parsing
% \newcommand{\lrhd}[1]{\mathrel{\mathord{\lhd}{^{#1}}\mathord{\rhd}}}
% \newcommand{\parse}[3]{#1 \, {\lrhd #2} \, #3 }
\newcommand{\parse}[3]{#1 \, {  \mathord{\lhd} {#2} \mathord{\rhd}  } \, #3 }
\newcommand{\parseFun}[1]{\left(\mathord{\lhd} {#1} \mathord{\rhd}\right)}
\newcommand{\parseFunE}{\mathfrak {parse } \,}                                                     % TODO !!
\newcommand{\print}[3]{#1 \, {  \mathord{\rhd} {#2} \mathord{\lhd}  } \, #3 }
\newcommand{\infru}[2]{\infer{#2}{#1}}
\newcommand{\infnamed}[3]{\infer[#1]{#3}{#2}}

\newcommand{\x}{X} % set of variables
\newcommand{\te}{T} % set of terms in basic syntax
\newcommand{\ot}{oT} % ordered terms
\newcommand{\otw}{oT_\Gamma} % ordered terms
\newcommand{\otc}{\overline{oT}} % closed ordered terms
\newcommand{\va}{V} % values
\newcommand{\xl}{\vec \x} % variable list
\newcommand{\vl}{\vec \va} % values list
\newcommand{\gl}{\vec \Gamma} % gamma list

% Printing / ``translating''
% \newcommand{\pr}[2]{ \prE  \hspace{-2pt} \left( #1 \ , \  #2 \right)}
\newcommand{\pr}[2]{ \prE   \left( #1 \ , \  #2 \right)}
\newcommand{\prE}{\lhd \hspace{-8pt} \lhd}
\newcommand{\prVal}[1]{\lhd \hspace{-2pt} \left( #1 \right)}
\newcommand{\prValE}{\mathord{\lhd}}
\newcommand{\prTree}[1]{\lhd \hspace{-8pt} \lhd \hspace{-8pt} \lhd \left( #1 \right)}
\newcommand{\prTreeE}{\lhd \hspace{-8pt} \lhd \hspace{-8pt} \lhd }

% one step reduction
\newcommand{\osr}{\rightarrow}
\newcommand{\tree}[2]{\langle \hspace{-2pt}\langle #1 \ , \  #2 \rangle \hspace{-2pt}\rangle}
%\newcommand{\tree}[2]{\left\langle \left\langle #1 \ , \  #2 \right\rangle \right\rangle}

\newcommand{\evTr}{\mathsf{evalTrees}}


\title{A Lambda Term Representation \\ Based on Linear Ordered Logic}
\author{Andreas Abel
\institute{
Theoretical Computer Science\\
Institut f\"ur Informatik\\
Ludwig-Maximilians-Universit\"at\\
M\"unchen, Germany}
\email{andreas.abel@ifi.lmu.de}
\and
Nicolai Kraus
\institute{
Functional Programming Laboratory\\
School of Computer Science\\
University of Nottingham\\
Nottingham, United Kingdom}
\email{ngk@cs.nott.ac.uk}
}
\def\titlerunning{A Lambda Term Representation Based on Linear Ordered Logic}
\def\authorrunning{Andreas Abel and Nicolai Kraus}
\begin{document}
\maketitle

\begin{abstract}
We introduce a new nameless representation of lambda terms based on
ordered logic.  At a lambda abstraction, number and relative
position of all occurrences of the bound variable are stored, and
application carries the additional information where to cut the
variable context into function and argument part.  This way, 
complete information about free variable occurrence is available at each
subterm without requiring a traversal, and environments can
be kept exact such that they only assign values to variables that
actually occur in the associated term.
Our approach avoids space leaks in interpreters that build 
function closures.  

In this article, we prove correctness of the new representation and
present an experimental evaluation of its performance in a proof
checker for the Edinburgh Logical Framework.
  
Keywords:
representation of binders,
explicit substitutions,
ordered contexts,
space leaks,
Logical Framework.

% We introduce a nameless representation of lambda terms based on ordered logic. 
% Information about number and places of variables bound by a lambda is available without examining the whole term, thus making it possible to drop unneeded substitutions early to avoid memory leaks.
% % We describe an implementation of this and other representations as well as suggested evaluation algorithms. These implementations were tested in Haskell by using them for typechecking large dependently typed terms of the logical framework (LF). The different needs of time and space are documented and compared.\\
% We also describe our implementation experiments and present some results.
\end{abstract}

\section{Introduction}
\label{sec:intro}

Type checking dependent types in languages like Agda~\cite{norell:PhD}
and Coq~\cite{inria:coq83} or logical frameworks like Twelf~\cite{carsten:twelf}
requires a large amount of evaluation, since types may depend on
values.  Such type checkers incorporate
an interpreter for purely functional programs with free variables---at least, the
$\lambda$-calculus---which is used to compute weak head normal forms
of types.  Efficiency of type checking is mostly identical with
efficiency of evaluation (and, in case of type reconstruction,
efficiency of unification), and remains a challenge as of today.  
In seminal work, Gregoire and Leroy \cite{gregoireLeroy:icfp02} have
sped up Coq type checking by compiling to byte-code instead of
using an interpreter.  Boespflug \cite{boespflug:padl10} has obtained
further speed-ups by producing native code using stock-compilers.  

While compilation approaches are successful on batch type \emph{checking}
fully explicit programs, they have not been attempted on type
\emph{reconstruction} using higher-order unification or on interactive
program construction such as in Agda and Epigram \cite{mcBrideMcKinna:view}.
These languages are involving and their implementations are
prototypical and frequently modified and extended.  Implementing a
full compiler just to get type reconstruction going is deterring;
furthermore, compilation has not proven its feasibility in minor
evaluation tasks that dominate higher-order unification.  Smart
interpreters are, and may remain competitive to compilation.  

For instance, Twelf's interpreter is sufficiently fast; it is 
based on a term representation with de Bruijn indices 
\cite{deBruijn:nameless} and explicit substitutions 
\cite{abadiCardelliCurienLevy:jfp91}.  In the context of functional
programming, explicit substitutions are known as \emph{closures},
consisting of the code of a function plus an environment, assigning
values to the free variables appearing in the code.  
In typical implementations of interpreters \cite{coquand:type}, 
these environments are not
precise; they assign values to all variables that are statically in
scope rather than only to those that are actually referred to in the
code.  This bears potential for space-leaks: the environment of a
closure might refer to a large value that is never used, but cannot be
garbage collected.  An obvious remedy to this threat is, when forming
a closure, to restrict the environment to the actual free variables;
however, this requires a traversal of the code.  We explore a
different direction: we are looking for a code representation that
maintains information about the free variables at each node of the
abstract syntax tree.

A principal candidate is \emph{linear typing} in Curry-Howard
correspondence with Girard's linear logic \cite{girard:linear}; there,
each variable in scope is actually referenced (more precisely,
referenced exactly once).  In other words, the
free variables are exactly the variables in the typing context.
Dropping types, we may talk of \emph{linear scoping}.  The free
variables of a function application are the disjoint union of the free
variables of the function and the free variables of the argument.  If
we want to maintain the set of free variables during a term traversal,
at an application node we need to decide which variables go into the
function part and which into the argument part.  Thus, we would store
at each application a set of variables that go into the, say, function
part, all others would go to the argument part.  Less information is
needed if we switch to an \emph{ordered} representation.

\emph{Ordered logic}, also called \emph{non-commutative linear logic}
\cite{polakowPfenning:tlca99}, refines linear logic by removing the
structural rule \emph{exchange} which restricts hypotheses to be used
\emph{in the order they have been declared}.  Transferring this
principle to ordered scoping this means that the scoping context lists
the free variables in the order they occur in the term, from left to
right.  This allows pushing the context into an application with very
little information: we just need to know \emph{how many} variables
appear in the function part so we can cut the context in two at the
right position, splitting it into function context and argument
context.  This constitutes the central idea of our representation: at
each application node of the syntax tree, we store a number denoting
the number of free variable occurrences in the function part.  During
evaluation of an application in an environment, we can cut the
environment into two, the environment needed for the evaluation of the
function and the environment needed for the evaluation of the
argument.  Thus, our environments are precise and space leaks are
avoided.  In particular, a variable is always evaluated in a singleton
environment assigning only a value to this variable.  Following this
observation, variables do not need a name, they are identified by
their position; and environments are simply sequences of values.

Since we are not interested in proof terms of ordered logic per se,
but only borrow the \emph{ordered context} idea for our
representation, we need to allow multiple occurrences of the same
variable.  In fact, the context shall list the variable
\emph{occurrences} in order.  At a lambda abstraction, we bind all
occurrences of the same variable.  Thus, at an abstraction node we
specify at which positions the bound variable should be inserted in
the scoping context.  This concludes the presentation of our idea.  

In the rest of the paper, after an introductory example
(Section~\ref{sec:example}) we formally define our term representation
in Section~\ref{sec:syntax}.   Interpreter and handling of
environments are described in Section~\ref{sec:values}, followed by
the translation between ordinary lambda terms and ordered terms
(Section~\ref{sec:parsing}).  Soundness of the interpreter is formally
proven in Section~\ref{sec:sound}, before we conclude with an
experimental evaluation in Section~\ref{sec:experiments}.

This article summarizes the B.Sc.\ thesis of the second author
\cite{kraus:bachelor}.

\section{An Example}
\label{sec:example}

% Beta reduction is an important part of evaluation in functional
% programming languages, proof assistants and other formal
% systems. However, implementations often involve the risk of memory
% leaks and inefficiency. Here, we assume that all of our terms are
% well-typed and beta reduction is therefore strongly normalizing, so
% all we care about is efficiency.


%To demonstrate the problem we want to deal with, we apply the term we just mentioned on the free variables $g$ and $f$, consecutively.
%To get rid of the beta redexes, we could reduce it in the way shown below. By writing $t\multisubs {s_1} {x_1} {s_n} {x_n}$, we want to express that in the term $t$, each occurrence of the variable $x_1$ ($x_2, \ldots, x_n$) has to be replaced by the term $s_1$ (resp. $s_2, \ldots, s_n$) simultaneously. Such a substitution list always applies only to the directly preceding term:

To demonstrate the discussed risk of space-leaks during evaluation, we apply the term $\LaA x {\LaA y {\ApA{\ApA a b} y}}$ in basic syntax consecutively on the free variables $g$ and $f$. A possible (and, if the mentioned closures are used, typical) sequence of reduction steps is given below.  
By writing $t\multisubs {s_1} {x_1} {s_n} {x_n}$, we want to express that in the term $t$, each occurrence of the variable $x_1$ ($x_2, \ldots, x_n$) has to be replaced by the term $s_1$ (resp. $s_2, \ldots, s_n$) simultaneously. Such a substitution list always applies only to the directly preceding term:
\[
\begin{array}{lr@{}l}
\shspace &(\LaA x {\LaA y {\ApA{\ApA a b} y}}) & \sspace g \sspace f \\ 
\shspace\re &(\LaA y {\ApA{\ApA a b} y}) & \sub g x \lspace f \\ 
\shspace\re &      (\ApA{\ApA a b} y) & \subs g x f y \\ 
\shspace\re &      \multicolumn 2 l {(a\sspace b)\subs g x f y \lspace y\subs g x f y} \\ 
\shspace\re &      \multicolumn 2 l {a\subs g x f y \lspace b\subs g x f y \lspace f} \\ 
\shspace\re &      \multicolumn 2 {c} {a \sspace b \sspace f}  \\ 
\end{array}
\]
Here, the substitution $\sub g x$ could be dropped instantly and there is no need to apply the other substitution $\sub f y$ to the term $\ApA a b$. 
However, the term representation used above comes along with the problem that such an evaluation algorithm does not have the required information in time.
This is due to the fact that the binding information is always split between the $\la$ and the actual variable occurrence, as they both carry the variable name.  
In contrast, using de Bruijn indices would make it possible to remove the piece of information from the $\la$.
Our goal is to do it the other way round: We want the whole information to be available at the $\la$, thus making it possible to know the number and places of the bound variable occurrences without looking at the whole term.


%A term representation always has to have a way to represent that a variable is bound by a lambda. 
%For example, in the term $\LaA x {\LaA y {\ApA{\ApA a b} y}}$ in basic syntax, the information about the connection between the second $\la$ and the last variable is split: We have to look at both the $\la y$ and the $y$ in order to know what is going on. This may lead to some unnecessary difficulties. In contrast, the usage of de Bruijn indices makes it possible to remove the information from the $\la$ and concentrate them on the variable. Our goal is to do it the other way round: We want the whole information to be available at the $\la$.


% However, the algorithm used above does not have this information in time and this is a problem that many representations come along with. Motivated by this observation, we want to describe an alternative.

\section{Syntax}
\label{sec:syntax}

% The idea is that the information which variable is bound by which lambda should neither be split (like in the standard representation) nor carried by the variable (de Bruijn indices). Instead, it should be available as soon as the lambda becomes visible.
In this article, we only cover the core constructs of the lambda calculus as they are enough to make the approach clear. However, we do not see any limitations for common extensions. We first define \emph{ordered preterms}:
\[
\begin{array}{lllrll@{\qquad}}
\mathsf{ordered \ preterms}       & \ni & t,u & ::= & x & \mbox{free variable (named $x$)} \\
			                   &&& \mid & \ovar & \mbox{bound variable (nameless)} \\
			                   &&& \mid & \ApO t m u & \mbox{application} \\
			                   &&& \mid & \LaO {\vec k} t & \mbox{abstraction} \\
\end{array}
\]
\emph{Free variables} are denoted by their name like in the standard syntax. \emph{Bound variables}, however, are just denoted by a dot $\ovar$, which does not carry any information beside the fact that it is a bound variable. 
% This, for example, means that if $\left(\ApO \ovar 1 \ovar \right)$ is a subterm, it is not possible to tell whether we have two different variables or two times the same one without looking at the whole term.\footnote{However, the latter possibility could not be well-typed.} 
In the case of an \emph{application}, there is a first term (the function part) and a second term (the argument) as usual. 
Furthermore, the application carries an integer $m$ as an additional piece of information that will be important for the evaluation process and is explained in a later paragraph.
%the number of dots $\ovar$ in the first term which are not bound by lambdas in the first term itself should be denoted by the integer $k$. % COMMENT - there is no reason anymore to prefer the second to the first term. It might be a good idea to use the first term instead!? I have changed it now.
%This will be important for the evaluation process. 
The most interesting part is the \emph{abstraction} $\LaO {\vec k} t$. The vector $\vec k = \ve{k_1, k_2, \ldots, k_n}$ is nothing else but a list of non-negative  integers of length $n$. It determines which dots $\ovar$ are bound by the $\la$ in the following way: Consider all $\ovar$ in the term $t$ which are not bound in $t$ itself. Now, the first $k_1$ of these are not bound by the $\la$, the next one is, the following $k_2$ are again not bound and so on (see examples below). 

We denote the number of unbound $\ovar$ in an ordered preterm $t$ by $\freevars t$. Consequently, $\freevars \cdot$ is simply defined by: 
\[
\begin{array}{cccc}
\freevars x = 0, & \freevars \ovar = 1, & \freevars {\ApO t m u} = \freevars {t} + \freevars{u}, & \freevars {\LaO {\ve {k_1, \ldots, k_n}} t} = \freevars t - n.
\end{array}
\]
%Note that in the case of $\ApO t k u$, we have $k = \freevars u$ by definition of $k$. \\
Any ordered preterm $u$ is an \emph{ordered term} if each lambda abstraction $\LaO {\ve {k_1, \ldots, k_n}} t$ which is a sub-preterm of $u$ satisfies the condition $n + \sum_i k_i \leq \freevars t$. In other words, if a $\la$ in $u$ binds a variable, this variable must actually exist. In addition, each sub-preterm $\ApO t m u$ has to fulfil the condition $m = \freevars t$.
$t$ is called \emph{closed} if $\freevars t = 0$.
% From now on, we only handle terms and forget about invalid preterms.

Here are some examples of closed terms.
The $S$ combinator 
\[
\LaA x {\LaA y {\LaA z {\ApA x z \sspace \pa {\ApA y z}}}}
\]
would be written as 
\[\LaO {\ve 0} {\LaO {\ve 1} {\LaO {\ve{1,1}} {\ApO {\ApO \ovar 1 \ovar} 2 {\pa{\ApO \ovar 1 \ovar}}}}}
\]
Moreover, the term 
\[
(\LaA x {\LaA y {\ApA{\ApA a b} y}}) \sspace g \sspace f
\]
from Section~\ref{sec:example} would be represented as
\[
\ApO{\left(\ApO {\LaO{\emptyVec}{\LaO{\ve 0}{\ApO{\ApO{a}{0}{b}}{0}{\ovar}}}\right)}  0 f } 0 g
\]
(note that applications are still left-associative). We can see that the first $\la$ does not bind anything as it is annotated with the empty vector $\emptyVec$, while this is less obvious when it is written as $\la x$. 

At this point, we hope to have clarified the intended meaning of our syntax. A formal definition will be given in Section~\ref{sec:parsing}.


\section{Values and Evaluation} 
\label{sec:values}

Before specifying values and evaluation formally, we want to give an example to demonstrate how the information carried by a lambda should be used and why we always have exactly the needed information. Suppose we have the term
\[
\left(\LaO {\ve 0} {\LaO {\ve 1} {\LaO {\ve{1,1}} {\ovar \oapp 1 \ovar \oapp 2 {\pa{\ovar \oapp 1 \ovar}}}}}\right) \sspace g \sspace f \sspace n 
%\left(\LaO {\ve 0} {\LaO {\ve 1} {\LaO {\ve{1,1}} {\ovar \oapp 1 \ovar \oapp 2 {\pa{\ovar \oapp 1 \ovar}}}}}\right) \oapp 0 g \oapp 0 f \oapp 0 n, \\ 
\]
that is, the $S$ combinator applied to three free variables
% i.e. the $S$ combinator which is first applied to a free variable $g$, then to $f$ and finally to $n$ 
(we suppress the application indices $0$ for better readability). We want to get rid of the beta redexes, so we start by eliminating the first one. The outermost $\la$ is decorated with the vector $\ve 0$ of length one. Now, the single variable bound by this $\la$ should be replaced by $g$, so we start a substitution list and insert a single $g$:
\[
\left(\LaO {\ve 1} {\LaO {\ve{1,1}} {\ovar \oapp 1 \ovar \oapp 2 {\pa{\ovar \oapp 1 \ovar}}}}\right) \left[ g \right] \lspace f \sspace n
\]
The first remaining $\la$ is $\la^{\ve 1}$, so it does not bind the first variable (thus $g$ remains first in the substitution list), but the second one. Consequently, we add an $f$ after the $g$:
\[
\left(\LaO {\ve{1,1}} {\ovar \oapp 1 \ovar \oapp 2 {\pa{\ovar \oapp 1 \ovar}}}\right)  \left[ g, f \right] \lspace n
\]
Now the situation becomes more interesting. The only remaining $\la$ is now decorated with the vector $\ve{1,1}$, so one $n$ has to be inserted after the first entry ($g$) and another one must be placed after the following entry ($f$):
\[
\left(\ovar \oapp 1 \ovar \oapp 2 {\pa{\ovar \oapp 1 \ovar}}\right)  \left[ g, n, f, n \right]
\]
All $\la$ have now been eliminated. The applications' indices tell us how the substitution list should be divided between the terms:
\[
\left(\ovar \oapp 1 \ovar\right)\left[ g, n \right] \lspace \left(\ovar \oapp 1 \ovar\right) \left[ f, n \right]
% \left(\ovar \oapp 1 \ovar\right)\left[ g, n \right] \lspace \oapp 0 \lspace \left(\ovar \oapp 1 \ovar\right) \left[ f, n \right]
\]
We do the same step once more:
\[
\ovar[g]  \lspace \ovar[n]  \lspace \left(\ovar[f]  \lspace \ovar[n]\right)
\]
The only thing left to be done is to apply the substitutions in the obvious way:
\[
g \sspace n  \sspace (f \sspace n)
\]
Here, evaluation naturally leads to a term in beta normal form. This is not always the case: as an example, if we had tried to evaluate the above term without the $n$ (i.e. $S \oapp 0 f \oapp 0 g$), we would have got stuck at $\left(\LaO {\ve{1,1}} {\ovar \oapp 1 \ovar \oapp 2 {\pa{\ovar \oapp 1 \ovar}}}\right)  \left[ g, f \right]$. However, this would have been satisfactory as it would have shown that the term's normal form is an abstraction. In other words, our evaluation results in weak head normal forms.
\\
Consequently, we define values in the following way:
\[ 
\begin{array}{lllrll@{\qquad}}
\mathsf{values}       & \ni & v,w & ::= & x \sspace \vec v & \mbox{large application} \\ 
% \noalign{\medskip}
			                   &&& \mid & \Val {\vec k} t {\vec v} & \mbox{closure} \\ 
\end{array}
\]
% \[
% \mathsf{values} \ni v, w ::= x \, \vec v \ \mid \ \ClosO{\vec k}{t}{}{\vec v}
% \]
The \emph{large application} consists of a variable $x$ which is applied to a vector $\ve{v_1, v_2, \ldots, v_m}$ of values. It is to be read as a left-associative application, i.e. as $\pa{\pa{x \, v_1} v_2 \ldots } v_m$. 
Note that it is not necessarily ``large''. Quite the contrary, it often only consists of the head (and the vector of values is empty).\\
A \emph{closure} $\Val {\vec k} t {\vec v}$ is the result of the evaluation process if the corresponding beta normal form of the term does not start with a free variable. The main part, $\LaO {\vec k} t$, is nothing other than a lambda abstraction in the syntax of ordered terms.
In addition, we need the substitution list $\vec v$ (which is simply a list of values) that satisfies $\length {\vec v} = \freevars{\LaO {\vec k} t}$. % as there might be unbound $\ovar$ in $\LaO {\vec k} t$. 
The idea is that the $i^{th}$ unbound $\ovar$ is to be replaced by $v_i$.  These substitution lists have already been used in the example above.


At this point, we want to introduce a notation for inserting a single item multiple times into a list. More precisely, if $\vec v = [v_1, v_2, \ldots, v_m]$ is a list, 
%(e.g. of values), % TODO neil suggests to delete this - I am not yet sure about it
$\vec k = [k_1, k_2, \ldots, k_n]$ is a vector (i.e. also a list) of  non-negative integers satisfying $\sum_{i=1}^n k_i \leq m$ and $w$ is a single item, we write $\multiinsert {\vec v} {\vec k} {w}$  for the list that is constructed by inserting $w$ at each of the positions $k_1, k_1 + k_2, \ldots, \sum_{i=1}^n k_i$ into $\vec v$, i.e. for the list $[v_1, v_2, \ldots, v_{k_1}, w, v_{k_1 + 1}, \ldots, v_{k_1 + k_2}, w, v_{k_1 + k_2 + 1}, \ldots, v_m]$ (of course, it is possible that $\multiinsert {\vec v} {\vec k} {w}$ starts or ends with $w$).


We are now able to define the evaluation function $\ev {\cdot} {\cdot}$ which takes an ordered term $t$ as well as an ordered substitution list $\vec v$ and returns a value. The tuple must always satisfy the condition $\freevars t = \length {\vec v}$. In other words, the list carries neither too little nor redundant information.
At the start of the evaluation, the ordered substitution list is empty. 
Additionally, we specify the application $\cdot \ap \cdot$ of two values, which also returns a value and does not need anything else.

\[ 
\begin{array}{lcl}
 \ev x {\emptyVec} & = & x \\ \noalign{\medskip}
 \ev \ovar {\ve{v_1}} & = & v_1 \\  \noalign{\medskip}
 \ev {t \oapp k u} {\ve{v_1, \ldots, v_n}} & = & {\ev t {\ve{v_1, \ldots, v_k}}} \ap {\ev u {\ve{v_{k+1}, \ldots, v_n}}} \\  \noalign{\medskip}
 \ev {\LaO {\vec k} t}  {\vec v} & = & \Val {\vec k} t {\vec v} \\  
\\ 
 \pa{x \sspace \vec v} \ap w & = & x \sspace \ve{\vec v, w} 
\\ \noalign{\medskip}
 \Val{\vec k} t {\vec v}   \ap w & = & \ev t {\multiinsert{\vec v}{\vec k}{w}}
\end{array}
\]
First, if we want to evaluate a free variable, the substitution list must be empty because of the invariant mentioned above. 
Second, in the case of a $\ovar$, the ordered list must have exactly one entry. This entry is the result of the evaluation. 
If we evaluate an application, we evaluate the left and the right term. The application's index enables us to split the substitution list at the right position. Then, we have to apply the first result to the second. 
Evaluating an abstraction is easy. We just need to keep the substitutions to build a closure. 

If we want to apply a large application to a value $w$, we just append $w$ to the vector of values (we write $\ve{\vec v, w}$ for $\ve{v_1, \ldots, v_n, w}$). 
The case of a closure $\Val {\vec k} t {\vec v}$ is less simple, but it is still quite clear what to do: $\vec k$ determines at which positions $w$ should be inserted in the ordered substitution list, so we just construct the list $\multiinsert {\vec v} {\vec k} w$. Then, $t$ is evaluated.


Concerning substitution lists, we talk about ``lists of values'' for simplicity. More specifically, we want them to be lists of pointers to avoid the duplication of  ``real'' values during constructing lists like $\multiinsert {\vec v} {\vec k} {w}$. They could also be implemented as binary trees instead of linked lists, cf. Section~\ref{sec:experiments}.


\section{Parsing and Printing}
\label{sec:parsing}

In this section, we define how terms in normal syntax are translated into our ordered syntax (Parsing) and vice versa (Printing). 
To specify this, we need some notation. 
First of all, we write $\x$ for the set of variable names we want to use and $\te$ for the set of lambda terms in basic standard syntax (i.e. $\x \subset \te$, furthermore, $x \in \x$ together with $t, u \in \te$ implies $\ApA t u \in \te$ and $\LaA x t \in \te$). Additionally, $\ot$ is the set of terms in our ordered syntax defined above, $\otc$ the subset of closed ordered terms ($\freevars t = 0$) and $\va$ the set of values (defined in the previous section).
Moreover, we write $\xl$ for the set of lists of variable names and $\vl$ for the set of lists of values. 
For each set $\Gamma$ of variable names ($\Gamma \subseteq \x$), we denote the set of lists of elements of $\Gamma$ by $\gl$ and the ordered terms that do not contain any variable of $\Gamma$ (as a free variable) by $\otw$.
% We will often allow $\Gamma$ to be a finite multiset, i.e. a finite set that can contain each element more than once. The according notion of union $\uplus$ satisfies $\Gamma \uplus \{z\} \not= \Gamma$. 

By writing $\ot \otimes \xl$ (resp. $\ot \otimes \vl$, $\otw \otimes \gl$, $\ldots$), we mean the subset $\{(t, \vec x) \ | \ \freevars t = \length {\vec x} \}$ of $\ot \times \xl$ (and analogous for the other cases).

% \begin{defin}
For each finite %multiset 
set $\Gamma$ of variable names, we define the relation
% $\parse \cdot \Gamma \cdot \ \subset \  \te \times (\ot \times \xl)$.
$\parseb \cdot \Gamma \cdot \cdot ~ \subset ~ \te \times \ot \times \xl$.
The intuition is that $\parseb M \Gamma u {\vec x}$ means: 
% The intuition is that $\parse M \Gamma (u, \vec x)$ (or simply $\parse
% M \Gamma {u \, \vec x}$) means: 
$M$ is a term that corresponds to the ordered term $u$, where unbound $\ovar$ are replaced by the variables in the list $\vec x$. 
The set $\Gamma$ can be seen as a filter that tells us which free variables do not occur in $u$ but in $\vec x$ instead.
\[
\begin{array}{c@{\qquad\qquad} c}
\infru{x \in \Gamma}{\parseb x \Gamma \ovar x}
&
\infru{ {\parseb M \Gamma t {\vec x}} \qquad 
        {\parseb N \Gamma u {\vec y}} \qquad {\length{\vec x} = m} } 
      {\parseb {M\,N} {\Gamma} {(\ApO t m u)}  {\vec x, \vec y} } 
\\ \\
\infru{x \not\in \Gamma}{\parseb x \Gamma x {}}
&
\infru{ 
         {\parseb M { (\Gamma \cup \{z\}) }  t {\multiinsert{\vec x}{\vec k}{z}   }} 
          \qquad 
          z \not\in \vec x  }
          {\parseb {\LaA z M} {\Gamma} {(\LaO {\vec{k}} t)} {\vec x}}
\end{array}
\]
%\end{defin}
It is important to note that $\parseb M \Gamma u {\vec x}$ implies that each variable occurring in $\vec x$ is contained in $\Gamma$ and each free variable occurring in $u$ is not contained in $\Gamma$. This can be shown by induction on $M$ (simultaneously for all sets $\Gamma$). 

By the same argument, one can see that (for each $M$ and $\Gamma$) there exists a unique tuple $(u, \vec x)$ satisfying $\parseb M \Gamma u  {\vec x}$, so we can consider $\parseFun \Gamma$ a function $\te \rightarrow \otw \times \gl$. Furthermore, we note that $\parseb M \Gamma u {\vec x}$ always implies $\freevars u = \length {\vec x}$. 

This also works the other way round. For each $\Gamma$ and each tuple $(u, \vec x) \in \otw \otimes \gl$, there is (by induction on $u$) a term $M \in \te$ satisfying $\parseb M \Gamma u {\vec x}$. Moreover, this term $M$ is unique up to alpha equivalence. So, $\parseFun \Gamma$ is actually a bijection between $\te / \alpha$ (the set of $\alpha$ equivalence classes of terms) and $\otw \otimes \gl$. The inference rules above show how to apply this bijection or its inverse to a term or a tuple (in the last rule, any variable satisfying the condition can be chosen for $z$), so we have (computable) functions $\parseFun \Gamma : \te / \alpha \rightarrow \otw \otimes \gl$ and $\parseFun \Gamma ^{-1}$. 

Choosing $\Gamma = \emptySet$, we get bijections $\te \leftrightarrow \ot \otimes {\vec \emptySet}$. As $\vec \emptySet$ is only inhabited by the empty vector $\emptyVec$, we naturally get the 
function $\parseFunE$ which maps $\te$ bijectively on $\otc$.


%Choosing $\Gamma = \emptySet$, we get the special cases $\parseFun \emptySet$ and $\parseFun \emptySet ^{-1}$, which are bijections between $\te$ and $\otc$. This also explaines how one representation can be transformed into the other.

The above construction also gives us a function $\otc \rightarrow \te$, but this is not enough. We want to transform the (more general) elements of $\ot \otimes \vl$ or just $\va$ into basic terms $\te$.
Therefore, we
% \begin{defin}
% We 
define the two printing functions $\prE : \ot \otimes \vl \rightarrow \te / \alpha$ and $\prValE : \va \rightarrow \te / \alpha$ simultaneously by recursion on the structure:
\[
\begin{array}{lcl}
\medskip
\pr{x}{\emptyVec} & = & x \\
\medskip
\pr{\ovar}{[v]} & = & \prVal v \\
\pr{t \oapp m u}{\vec v} & = & \ApA {\pr {t}{\vec{v_{start}}}  }  {\quad \pr {u}{\vec{v_{rest}}}} \\
\medskip
&& \mbox{(split $\vec v$ at position $m$ to get $\vec{v_{start}}$ and $\vec{v_{rest}}$)} \\
% \pr{\LaO {k_1, \ldots, k_m} t}{[v_1, v_2, \ldots, v_n]} & = & \LaA z {\pr {t} {v_1, \ldots, v_{k_1}, z, v_{k_1+1}, \ldots, }}   \\
\pr{\LaO {\vec k} t}{\vec v} & = & \LaA z {\pr {t} {  \multiinsert{\vec v}{\vec k}{z}   }}   \\
\medskip
&& \mbox{where $z$ is any variable that does not occur freely in $t$ or $\vec v$ } \\
\prVal{x \sspace v_1 \sspace v_2 \sspace \ldots \sspace v_n} & = & x \ \prVal{v_1} \ \prVal{v_2} \ \ldots \ \prVal{v_n} \\
\medskip && \mbox{(a large application simply becomes an application of terms)} \\
\prVal{\Val {\vec k} t {\vec v}} & = & \pr{\LaO {\vec k} t}{\vec v} \\
\end{array}
\]
%\end{defin}
First, note that the printing functions are well-defined (i.e. they always terminate). This is because during evaluation of $\pr{u}{\vec v}$, we may safely assume that $\pr{t}{\vec w}$ is well-defined as long as $t$ is a strict subterm of $u$ and each value $w'$ in $\vec w$ is either only a variable (so termination of $\prVal {w'}$ is clear) or also in $\vec v$. Similar, during evaluation of $\prVal {x \sspace v_1 \sspace \ldots v_n}$, we may assume that $\prVal {v_i}$ is defined for each $i$.

For all $t \in \otc, M \in \te$, we have $\pr t \emptyVec = M$ if and only if $\parseb M \emptySet t \emptyVec$ (which just means $\parseFunE t = M$) as both judgements are defined identically in the case of closed ordered terms. This essentially (with implicit use of the bijection $\otc \leftrightarrow \ot \otimes \vec \emptySet$) means $\prE \circ \parseFunE  = id_{\te}$, i.e. the composition of parsing and printing is the identity.






\section{Correctness and Termination properties}
\label{sec:sound}



We still have not shown that our evaluation algorithm given in Section~\ref{sec:values} does not change the meaning of terms. The combination of parsing, evaluating and printing should never result in a term that is not beta equivalent to the original term. We also want to show a limited termination property.
To keep our argument simple, we just sketch the proofs and hope that the ideas become clear.

First, we attend to the correctness question. 
We need to convince ourselves that \emph{rewriting} according to the rules of the functions $\ev \cdot \cdot$ and $\ap$ does not cause an error. 
By \emph{rewriting}, we mean one step of the \emph{normal} or \emph{leftmost outermost} evaluation. We have demonstrated this in the example at the beginning of Section~\ref{sec:values}.
%If we have, e.g., the tupel $(t, \vec v)$, printing it should result in a term that is $\beta$ equivalent to the term we get if we rewrite it once before printing. 
Printing should result in a term that is $\beta$ equivalent to the term we get if we rewrite before printing. 
This basically means that, for each evaluation rule on the left hand side of the following table, we have to check that the equality on the right hand side holds:
\begin{center}
\begin{tabular}{|   rcl  |  rcl    |}
\hline
&&&&& \\
$\ev x {\emptyVec} $&$=$&$ x    $ & $ \pr{x}{\emptyVec} $&$\, =_\beta \, $&$\prVal{x}$\\
&&&&& \\
$\ev \ovar {\ve{v}} $&$=$&$ v   $ & $ \pr{\ovar}{v} $&$\, =_\beta \,$&$ \prVal v$ \\
&&&&& \\
$\ev {t \oapp k u} {[\vec v , \vec w]} $&$=$&$ {\ev t {\vec v}} \ap {\ev u {\vec w}}   $ & $ \pr {t \oapp k u} {[\vec v , \vec w]} $&$\, =_\beta \,$&$  \ApA {\pr{t}{\vec v}} {\pr{u}{\vec w}} $\\
&&&&& \\
$\ev {\LaO {\vec k} t}  {\vec v} $&$=$&$ \Val {\vec k} t {\vec v} $ & $  \pr{\LaO {\vec k} t}{\vec v}$&$ \, =_\beta \,$&$  \prVal {\Val {\vec k} t {\vec v}} $ \\  
&&&&& \\
$\pa{x \sspace \vec v} \ap w $&$=$&$ x \sspace \ve{\vec v, w}   $ & $\ApA {\prVal {x \sspace \vec v} } {\prVal w}   $&$\, =_\beta \, $&$ \prVal{x \sspace \ve{\vec v, w}}$\\
&&&&& \\
$\Val{\vec k} t {\vec v}   \ap w $&$=$&$ \ev t {\multiinsert{\vec v}{\vec k}{w}}   $ & $ \ApA{\prVal{\Val{\vec k} t {\vec v}}}{\prVal w}  $&$ \, =_\beta \, $&$  \pr{t}{\multiinsert{\vec v}{\vec k}{w}}  $ \\
&&&&& \\
\hline
\end{tabular}
\end{center}
Note that ``rewriting using the evaluation rules'' results in expressions which are, more or less, a mixture of elements of $\ot \otimes \vl$ and $\va$. To be precise, such an expression is either in $\ot \otimes \vl$ or in $\va$ or a tuple (to read as simple application) of two expressions. The $1^{st}$, $2^{nd}$ and $4^{th}$ rule turn an element of $\ot \otimes \va$ into a value, the $3^{rd}$ turns it in a tuple of two such elements, and the last two rules turn a tuple of two values into one value or element. 
Although we do not define it formally, it should be clear how those expressions can be printed by using the printing functions for ordered terms and values (if an expression is a tuple, just print the function part and the argument part separately). In fact, while the function $\ev \cdot \cdot$ is formulated as a big step evaluation, the rewriting process can be understood as the corresponding small step (or one step) evaluation.

In the first five rows of the table, we only have to look at the definitions of the printing functions to see that the printed terms are not only $\beta$ equivalent but also equal. The very last rule requires closer examination: 
\\
By definition, the term $\prVal{\Val{\vec k} t {\vec v}}$ is equal to $\LaA z {\pr {t} {  \multiinsert{\vec v}{\vec k}{z}   }} $ for a (sufficiently) fresh variable $z$. Now, the definitions of the printing functions are ``context free'' in a way that guarantees  that $z$ occurs exactly $\length{\vec k}$ times (free) in $\pr {t} {  \multiinsert{\vec v}{\vec k}{z}   }$. Furthermore, replacing those occurrences by $\prVal{w}$ results in the term $\pr{t}{\multiinsert{\vec v}{\vec k}{w}}$. This means that, starting with 
$ \ApA{\prVal{\Val{\vec k} t {\vec v}}}{\prVal w}  $,
we have to use exactly one $\beta$ reduction step to get the term $  \pr{t}{\multiinsert{\vec v}{\vec k}{w}}  $.

As we have already seen that the composition of parsing a term and printing it afterwards does not change anything (up to $\alpha$ equivalence), we can now conclude that parsing, evaluating (a finite number of rewriting steps) and printing is equivalent to a number of $\beta$ reduction steps.

Now we discuss termination. Obviously, our evaluation function $\ev \cdot \cdot$ does not always terminate as some terms do not have a weak head normal form. However, $\ev \cdot \cdot$ terminates whenever it is applied to ($\parseFunE t$) if the usual $\beta$ reduction is strongly normalizing on $t$. 
The main consequence of this is that evaluation terminates for all well-typed terms. % TODO "all" !
To prove this statement, assume that there is such a term $s_0 \in \te$ so that the evaluation of $t_0 := \parseFunE s_0$ does not terminate. 
%
Then, we get an infinite sequence $t_0, t_1, t_2, \ldots$ where $t_{i+1}$ is the result of rewriting (a subexpression of) $t_i$ using one of the evaluation rules. If we print $t_0, t_1, t_2, \ldots$, we get a sequence $s_0, s_1, s_2, \ldots$ of terms in $\te$, where $s_{i+1}$ is either ($\alpha$) equal to $s_i$ or arises from $s_i$ in exactly one $\beta$ reduction step. If $\beta$ reduction is strongly normalizing on $s_0$, the sequence has to become constant at some point, i.e. $s_N = s_{N+1} = s_{N+2} = \ldots$ for some $N$. This implies that, after the first $N$ rewriting steps, the second $@$ rule (which corresponds to the last line in the table) is not used anymore. Define the \emph{weight} $w(t)$ of an expression $t$ to be $1$, if the expression is just an element of $\ot \otimes \va$, to be $2$, if it is a value of the \emph{closure} type, to be %TODO
$1 + 2^n + w(v_1) + w(v_2) + \ldots + w(v_n)$
, if it is a value of the form $x \, v_1 v_2 \ldots v_n$ (i.e. a \emph{large application}) and, if it is a tuple of two expressions, as the sum of both weights. Then, each of the rewritings that are induced by the first five lines in the table increase the weight of the expression, so we get $w(t_N) < w(t_{N+1}) < w(t_{N+2}) < \ldots$;  however, as the total number of values (and tuples in $\ot \otimes \va$) is bound by the length of the term we get after printing $t_N$ (or any $t_{N+i}$), the sequence is bounded, resulting in the required contradiction.



% -----------------------------------
%
% THIS IS AN OUTDATED VERSION OF THE SECTION. IT IS MORE "FORMALLY CORRECT" BUT CONTAINS TOO MUCH OVERHEAD AND IS THEREFORE DIFFICULT TO READ.
%
%We still have to prove two important facts about the evaluation algorithm given in section \ref{sec:values} - first, its termination properties and second, its correctness, i.e. it corresponds to beta reduction. 
%
%Concerning correctness, we would obviously want to show that taking a term $t$, translating it to an ordered term, evaluation and printing it gives us a term which is beta equal to $t$, which is represented by the diagram
%\[
%\begin{xy}
%\xymatrix{
% t \in \te \ar@{-}[rrr]^{=_\beta} \ar@/_/[dd]_{\parseFunE} & & & t' \in \te \\ 
% \\
%% u \in \otc \ar@/_/[uu]_{\pr{\cdot}{\emptyVec}} \ar[rrr]^{\ev{\cdot}{\emptyVec}} & & & \ev{u}{\emptyVec} \in \va \ar[uu]_{\prValE}
% u \in \otc \ar@/_/[uu]_{\prE                  } \ar[rrr]^{\ev{\cdot}{\emptyVec}} & & & \ev{u}{\emptyVec} \in \va \ar[uu]_{\prValE}
%}
%\end{xy}
%\]
%However, the diagram does not make much sense as we do not know yet that $\ev \cdot \emptyVec$ is a well-defined function.
%What we need to do is to define a one-step-evaluation, thereby making it possible to check if each single evaluation step preserves beta equality. Unfortunately, this becomes even more complicated as we get binary trees during evaluation: 
%\[
%\begin{array}{lllrll@{\qquad}}
%\evTr       & \ni & \mathsf{tree}_1, \mathsf{tree}_2 & ::= & (t, \vec v) & \mbox{a tupel in $\ot \otimes \vl$} \\
%			                   &&& \mid & v & \mbox{just a value} \\
%			                   &&& \mid & \tree {\mathsf{tree}_1} {\mathsf{tree}_2}  & \mbox{node with two subtrees} 
%\end{array}
%\]
%Our one-step reduction $\osr$ is a relation on $\evTr \times \evTr$:
%
%\begin{gather*}
%\infnamed{^{\ev \cdot \cdot \mbox{\small -1} }}{}{(x , \emptyVec) \osr x}
%\qquad
%\qquad
%\infnamed{^{\ev \cdot \cdot \mbox{\small -2}}}{}{(\ovar , \ve v) \osr v}
%\qquad
%\qquad
%\infnamed{^{\ev \cdot \cdot \mbox{\small -3}}}{}{(t \oapp k u , \vec v \vec w) \osr \tree {(t, \vec v)}{(u, \vec w)}}
%\\ \\
%\infnamed{^{\ev \cdot \cdot \mbox{\small -4}}}{}{(\LaO {\vec k} t , \vec v) \osr \Val {\vec k} t {\vec v}}
%\qquad
%\qquad
%\infnamed{^{\mbox{\small (struct-1)}}}{a \osr a'}{\tree a b \osr \tree {a'} b}
%\qquad
%\qquad
%\infnamed{^{\mbox{\small (struct-2)}}}{b \osr b'}{\tree a b \osr \tree a {b'}}
%\\ \\
%\infnamed{^{@\mbox{\small -1}}}{}{\tree {x \sspace v_1 \ldots v_n} w \osr  x \sspace v_1 \ldots v_n \sspace w}
%\qquad
%\qquad
%\infnamed{^{@\mbox{\small -2}}}{}{\tree {\Val {\vec k} t {\vec v}}{w}   \osr   (t, \multiinsert {\vec v} {\vec k} w)}
%\end{gather*}
%Now it is easy to define a printing function $\prTreeE$ to transform a tree into a term:
%\[
%\begin{array}{lclll}
%\prTree{(t, \vec v)} & = & \pr {t} {\vec v} &&\\
%\prTree{v} & = & \prVal v &&\\
%\prTree{\tree a b} & = & \ApA {\prTree a}{\prTree b}&& \mbox{(simply the application in $\te$)}
%\end{array}
%\]
%\begin{prop}
% For $a, b \in \evTr$ satisfying $a \osr b$, we have $\prTree a =_\beta \prTree b$. More precisely, if one of the rules ${\ev \cdot \cdot \mbox{-1} }, {\ev \cdot \cdot \mbox{-2} }, {\ev \cdot \cdot \mbox{-3} }, {\ev \cdot \cdot \mbox{-4} }$ or ${@\mbox{-1}}$ was used, even $\prTree a = \prTree b$ (up to $\alpha$-equivalence) holds, if ${@\mbox{-1}}$ was used,  $\prTree a$ can be reduced to $\prTree b$ in exactly one $\beta$-reduction step.
%\end{prop}
%\begin{proof}
% For all of the $\ev \cdot \cdot$-rules and $@\mbox{-1}$, the equality $\prTree a = \prTree b$ is easy to check. 
%% In the case of the structure rules, we may assume that the statement holds for the premises, so it is also clear for the conclusions. The first of the $@$-rules is also simple. 
% We have to have a closer look at the second $@$-rule:\\
% By definition, we have 
% \[
%\begin{array}{lclr}
% \prTree {\tree {\Val {\vec k} t {\vec v}}{w}} & = & \ApA {  \prVal {\Val {\vec k} t {\vec v}}   } \ {\prVal w}  & \\
%& = &\ApA {  \pr{\LaO {\vec k} t}{\vec v}   }  \ {\prVal w} \\
%& = & \ApA {\left( \LaA z {\pr {t} {\multiinsert {\vec v}{\vec k}{z}}} \right)} \ {\prVal w} &  \qquad \mbox{$z$ not free in $t$, $\vec v$}\\
%\mbox{and} &&& \\
% \prTree {    (t , \multiinsert{\vec v}{\vec k}{w} )    } & = & \pr{t}{\multiinsert{\vec v}{\vec k}{w}} &
%\end{array}
%\]
%Now, note that (again by induction on $t$) the term $\pr {t} {\multiinsert {\vec v}{\vec k}{z}}$ contains  $z$ exactly $\length {\vec k}$ times as a free variable and, by replacing each of those free occurrences  by $\prVal w$, we get the term $\pr{t}{\multiinsert{\vec v}{\vec k}{w}}$.
%Consequently, the term $\prTree {\tree {\Val {\vec k} t {\vec v}}{w}}$ can be reduced to $\prTree {    (t , \multiinsert{\vec v}{\vec k}{w} )    }$ in exactly one beta reduction step. \\
%Note that these statements are also true if one or more of the structure rules are used in addition.
%
%
%\end{proof}
%
%\begin{prop}
%For any well-typed term $t \in \te$, the One-Step-Reduction terminates for $(\parseFunE t)$.
%\end{prop}
%\begin{proof}
%Suppose there is an infinite sequence $a_0 \rightarrow a_1 \rightarrow a_2 \rightarrow \ldots$ with $a_0 = \parseFunE t$. According to the proposition above, we can conclude that for each $i$, either $\prTree {a_i} = \prTree {a_{i+1}}$ or $\prTree {a_i} \rightarrow_\beta \prTree {a_{i+1}}$ (in exactly one step). Assume the latter possibility occurs infinitely often. This contradicts the fact that $\beta$-reduction is strongly normalizing on well-types terms. Assume it occurs only finitely often. Then there is an $N$ such that $\prTree {a_N} = \prTree {a_{N+1}} = \prTree {a_{N+2}} = \ldots$. For $a \in \evTr$, we define the weight $w(a)$ as the number of leafs of $a$ containing an element of $\ot \otimes \vl$ plus twice the number of leafs that contain a value. % TODO: THIS HAS TO BE MODIFIED. (see newer version)
%Each of the one-step-rules except the last one (and the structure rules) obviously increases the weight of a tree, thus we get $w(a_N) < w(a_{N+1}) < w(a_{N+2}) < \ldots$. However, the printing function is defined in a way that forces the weight of any tree $a$ to be less or equal than twice the length of $a$ and consequently, the sequence above is bounded by $2 \cdot \length{ \prTree{a_N}}$, thus giving us a contradiction again.
%\end{proof}
%% TODO rename PARSINGfunction and make clear that sometimes parse: ... -> trees.
%
%Note that if the reduction terminates, the result is a tree that consists of exactly one leaf containing a value (as there is a reduction step for any other case). We identify this tree with the value.
%
%\begin{cor}
%For all well-typed terms, the evaluation function $\ev \cdot \cdot$ is well-defined (i.e. terminates) and correct (i.e. parsing a term, evaluating and printing it results in a $\beta$-equivalent term).
%\end{cor}
%\begin{proof}
%It is easy to check that the evaluation $\ev \cdot \cdot$ can be simulated by using a sequence of $\rightarrow$-steps. First, this shows that it terminates as we would get an infinite sequence of steps otherwise. Second, as each small step conserves $\beta$-equality, the whole sequence does the same. As we know that for $t \in \te$ the equality $\prTree {\parseFunE t , \emptyVec} = t$ holds, this implies $\prTree {\ev{\parseFunE t} {\emptyVec}} = t$. 
%\end{proof}
%
% -----------------------------------











\section{Experiments and Results}
\label{sec:experiments}

The specified term representation and evaluation have been implemented
in Haskell. They have been used by a type checker to check large files
of dependently
typed terms of the Edinburgh Logical Framework which were kindly
provided by Andrew W. Appel (Princeton University). To make this
possible, an extended syntax has been used that includes $\Pi$-types% and technical elements like
, constants and definitions. It is straightforward to expand our
evaluation algorithm to the extended syntax. The substitution lists
have been implemented as simple Haskell lists, and also as balanced
binary trees (following Adams \cite{adams:jfp93})
for better asymptotic complexity.  Both variants were
evaluated for performance 
[referred to as \emph{Ordered (trees)} and \emph{Ordered (lists)}].

For comparison, the completely analogous algorithm for terms in
extended basic syntax (i.e. $\te$) has been used [\emph{Simple
  Closures}]. Furthermore, we have tested a strategy that always
evaluates completely (i.e. produces $\beta$ normal forms) using
Hereditary Substitution [\emph{Beta Normal Values}].

Our main test file \textsf{w32\_sig\_semant.elf} with a size of
approximately 21 megabytes contains a proof described in
\cite{appel:toplas10}. We also tested smaller parts of this file, more
precisely, the first $6000$, $10,000$ and $12,000$ lines without the
rest (named \textsf{6000.elf} and so on). Later terms tend to be
larger, so the tests with fewer lines needed much less time.

All tests were executed on the same server 
\texttt{baerentatze.cip.ifi.lmu.de}
% ("baerentatze",
% Ludwig-Maximilians-Universit\"at M\"unchen, Department of Computer
% Science) 
working with a CPU of type \emph{AMD Phenom II X4 B95} (only
one core used, 3000 MHz) and 7999 MiB system memory. The measurements
of space and time consumption are given in the following tables (rounded
average values):

\begin{center}
\begin{tabular}{| l || c | c |}
\multicolumn{3}{c}{\textsf{6000.elf} (file size: 3.8 MB)}\\
\hline
& time (sec) & space (MB) \\
\hline
\hline
Ordered (trees) & 18.9 & 1111 \\
\hline
Ordered (lists) & 18.6 & 1114\\
\hline
Simple Closures & 18.5 & 1152\\
\hline
Beta Normal Values & 27.6 & 2034\\
\hline
\end{tabular}

\vspace{5pt}

\begin{tabular}{| l || c | c |}
\multicolumn{3}{c}{\textsf{10000.elf} (file size: 12.9 MB)}\\
\hline
& time (sec) & space (MB) \\
\hline
\hline
Ordered (trees) & 61.0 & 3230\\
\hline
Ordered (lists) & 60.6 & 3237\\
\hline
Simple Closures & 60.0 & 3302\\
\hline
Beta Normal Values & 98.7 & 5878\\
\hline
\end{tabular}

\vspace{5pt}

\begin{tabular}{| l || c | c |}
\multicolumn{3}{c}{\textsf{12000.elf} (file size: 17.8 MB)}\\
\hline
& time (sec) & space (MB) \\
\hline
\hline
Ordered (trees) & 84.3 & 5096 \\
\hline
Ordered (lists) & 83.8 & 5103 \\
\hline
Simple Closures & 83.6 & 5226 \\
\hline
Beta Normal Values & 137.7 & 8513 \\
\hline
\end{tabular}
\end{center}
Unsurprisingly, beta normal values perform significantly worse than each of the other possibilities. However, the difference is smaller than it could have been expected. This might be due to the fact that during type checking, total evaluation of a term is often necessary anyway, thereby reducing the hereditary substitution's disadvantage. \\
Although none of the other strategies exhibited any shortcomings in the comparisons above, the following results for the complete file are remarkable. Here, implementing ordered substitutions as normal Haskell lists seems to be much more efficient than using tree structures: 

\begin{center}
 \begin{tabular}{| l || c | c |}
\multicolumn{3}{c}{\textsf{w32\_sig\_semant.elf} (file size: 20.9 MB)}\\
\hline
& time (sec) & space (MB) \\
\hline
\hline
Ordered (trees) & 108.4 & 8877\\
\hline
Ordered (lists) & 94.8 & 4948\\ % average, 4 values.
\hline
Simple Closures & 94.3 & 5068 \\ % some more tests: 92.42 , 4994 (?!); 94.17, 5068; 94.97, 5068; 93.75, 4994
\hline
Beta Normal Values & 169.8 & 9044  \\
\hline
\end{tabular}
\end{center}
Our Simple Closures are still on the same level as Ordered Representation with lists, but the trees are far behind. 
In comparison, the type checker of the Twelf project, \emph{Twelf
  r1697} (written in \emph{Standard ML} and compiled with
\emph{MLton}'s whole program optimizations \cite{fluetWeeks:icfp01}) 
does the job nearly five
times faster while using only 2720 megabytes of
memory. % TODO only due to an optimized compiler?

%Unfortunately, the idea of ordered substitution lists does not really pay off within the context of type checking in our tests. 
%On the other hand, it can compete with the simple standard implementation. This offers hope that there might be some applications that benefit more from our suggested strategy. 




\section{Related Work and Conclusions}
\label{sec:concl}

Our term representation is inspired by intuitionistic implicational
linear logic in natural deduction style which has explicit operations
for weakening and contraction
\cite{bentonBiermanDePaivaHyland:tlca93}.  With explicit weakening and
contraction, one easily maintains complete information about the free
variables of a term at each node \cite{kesnerLengrand:infcomp07}.  Our
term representation incorporates weakening and contraction into lambda
abstraction.  By using inspiration from ordered logic, we reduce the
stored information at application nodes to a minimum, namely an
integer; further, our variable nodes need to carry no information at all.

Another means to maintain information about free variables are
\emph{director strings} by Sinot \cite{sinot:jlc05}.  Application
nodes come with a map that tell for each variable whether it appears
in the left or the right subterm or in both.  Our term representation
can be seen as an optimized version of director strings, however, we
have no experimental comparison. 
Sinot \etal~\cite{fernandezMackieSinot:aaecc05} present some
performance results of director strings; however, it is restricted
to evaluation of some specific big lambda-terms.  There is no study on
their relative performance in a realistic application---yet that is
our concern.

Liang, Nadathur, and Qi \cite{liangNadathurQi:jar05} have evaluated
different term representations in the context of $\lambda$Prolog, a
study that compares to our study of term representations for the
Edinburgh Logical Framework.  They have tested different combinations
of features, confirming our result that lazy substitution is 
preferable to eager substitution {[Beta Normal Forms]},
even more so when several substitutions are gathered into one
traversal {[Closures, Ordered]}.  They also test a variant where
each term is equipped with an \emph{annotation}, a flag telling whether
this term is open, \ie, has free variables, or closed, \ie, has no
free variables.  In their experimental evaluation, these annotations
pay off greatly for the poorly behaving eager substitution, yet give
negligible advantage for explicit substitutions.  It is
hypothesized that in a combined substitution, each subterm will
mention at least one variable with a high probability, so the
traversal has to run over most of the whole term---this is certainly
different in the substitution for a single variable.

To summarize, we have presented a new term representation for the
lambda-calculus based on ordered linear logic, and experimentally
compared it with well-known representations (closures, normal forms)
in a prototypical implementation of a type checker for the Edinburgh
Logical Framework.  The experiments were carried out on large realistic
proof terms, constructed manually and mechanically.  

The results were not significantly in favor of our new representation.
This might be due to the application domain, LF signature checking.
For one, LF-definitions are closed, which means that substitutions
never need to traverse a definition body when the definition is
expanded, and this optimization is shared by all the term
representations we compared.  Secondly, we only tested type checking,
not type reconstruction via unification.  During type checking, where
equality tests are expected to succeed, full normal forms are always
computed, and closures are very short-lived in memory.  More space
leaks are to be expected in applications such as logic programming or
type reconstruction, where unification is needed, which is not
expected to always succeed.   In constraint-based unification,
unsolvable constraints might be postponed, keeping closures alive for
longer.  In such situations, the benefits of our representation might
be more noticeable, more experiments are required.

In the future, we plan to investigate further term representations
such as term graphs, and perform more experiments.  The literature on
experimentally successful term representations is sparse, our work
contributes to close this gap.  Our long term goal is to find a
performing term representation to use for Agda's type reconstruction.


\paragraph*{Acknowledgements}
The idea for the here presented ordered term representation was
planted in a discussion with Christophe Raffalli who invited the first
author to Chambery in February 2010.  He mentioned to me that in his
library \texttt{bindlib} formation of closures restricts the
environment to the variables actually occurring free in the code.  I
also benefited from discussions with Brigitte Pientka and Stefan Monnier.

Thanks to Gabriel Scherer for comments on a draft version of this
paper and pointers to related work.

\bibliographystyle{alpha}
\bibliography{auto-lfmtp11}


\end{document}
